%========================================================================
%
%  PSE / Adjoint Notes
%
%  S. Collis
%
%  8-24-98
%========================================================================
\documentclass[12pt]{article}
\usepackage{fullpage}
\usepackage{cite}
\usepackage{xspace}
\usepackage{amsbsy}
\usepackage{subeqn}
\newif\ifCUPmtlplainloaded

%\usepackage{upmath}
%\usepackage{lucidbry}

%   EPSF.TEX macro file:
%
%===============================================================================
%   Special hacked version for my Ph.D. thesis
%
%   Scott Collis
%
%   3-10-97
%===============================================================================
%
%   Written by Tomas Rokicki of Radical Eye Software, 29 Mar 1989.
%   Revised by Don Knuth, 3 Jan 1990.
%   Revised by Tomas Rokicki to accept bounding boxes with no
%      space after the colon, 18 Jul 1990.
%
%   TeX macros to include an Encapsulated PostScript graphic.
%   Works by finding the bounding box comment,
%   calculating the correct scale values, and inserting a vbox
%   of the appropriate size at the current position in the TeX document.
%
%   To use with the center environment of LaTeX, preface the \epsffile
%   call with a \leavevmode.  (LaTeX should probably supply this itself
%   for the center environment.)
%
%   To use, simply say
%   \input epsf           % somewhere early on in your TeX file
%   \epsfbox{filename.ps} % where you want to insert a vbox for a figure
%
%   Alternatively, you can type
%
%   \epsfbox[0 0 30 50]{filename.ps} % to supply your own BB
%
%   which will not read in the file, and will instead use the bounding
%   box you specify.
%
%   The effect will be to typeset the figure as a TeX box, at the
%   point of your \epsfbox command. By default, the graphic will have its
%   `natural' width (namely the width of its bounding box, as described
%   in filename.ps). The TeX box will have depth zero.
%
%   You can enlarge or reduce the figure by saying
%     \epsfxsize=<dimen> \epsfbox{filename.ps}
%   (or
%     \epsfysize=<dimen> \epsfbox{filename.ps})
%   instead. Then the width of the TeX box will be \epsfxsize and its
%   height will be scaled proportionately (or the height will be
%   \epsfysize and its width will be scaled proportiontally).  The
%   width (and height) is restored to zero after each use.
%
%   A more general facility for sizing is available by defining the
%   \epsfsize macro.    Normally you can redefine this macro
%   to do almost anything.  The first parameter is the natural x size of
%   the PostScript graphic, the second parameter is the natural y size
%   of the PostScript graphic.  It must return the xsize to use, or 0 if
%   natural scaling is to be used.  Common uses include:
%
%      \epsfxsize  % just leave the old value alone
%      0pt         % use the natural sizes
%      #1          % use the natural sizes
%      \hsize      % scale to full width
%      0.5#1       % scale to 50% of natural size
%      \ifnum#1>\hsize\hsize\else#1\fi  % smaller of natural, hsize
%
%   If you want TeX to report the size of the figure (as a message
%   on your terminal when it processes each figure), say `\epsfverbosetrue'.
%==============================================================================
%
%  New stuff (SSC)
%
\def\includefigs{\let\ifincfigs=\iftrue}
\def\noincludefigs{\let\ifincfigs=\iffalse}
\includefigs

\newbox\epsfvertlab
\newbox\epsfhorlab
\newbox\epsffiglab

\newdimen\epsfvlabsize
\newdimen\scott
%
%  Define axis and figure labels
%
\def\setvlabel#1{\setbox\epsfvertlab=\vbox{\hbox{#1}}}%
\def\sethlabel#1{\setbox\epsfhorlab=\vbox{\hbox{#1}}}%
\def\figlab#1 #2 #3{\setbox\epsffiglab=\vbox to 0pt{%
\ifvoid\epsffiglab\else\box\epsffiglab\fi\vss\hbox to 0pt{\raise #2 \hbox{\hskip #1 #3}\hss}}}
%
%  Verttex macros
%
\newdimen\fighor
\newdimen\figver
\newbox\rotbox
\long\def\lrlap#1{\hbox to 0pt{#1\hss}}
\long\def\verttex#1#2#3{{\fighor = #1\figver = #2\vbox to \figver{\vss%
\hbox to \fighor{\hfill\hsize=\fighor%
\lrlap{\rotstart{-90 rotate}\vbox to \fighor{#3\vfil}\rotfinish}}}}}
%
%  These three macros are needed to implement verttex
%
\def\dvipsvspec#1{\special{ps:#1}}%  passes #1 verbatim to the output
\def\dvipsrotstart#1{\dvipsvspec{gsave currentpoint currentpoint translate
   #1 neg exch neg exch translate}}% #1 can be any origin-fixing transformation
\def\dvipsrotfinish{\dvipsvspec{currentpoint grestore moveto}}% gets back in synch
%
\def\rotstart#1{\dvipsrotstart{#1}}
\def\rotfinish{\dvipsrotfinish}
%
%  Place the axis labels
%
\def\epsfsetlab{%
\ifvoid\epsfvertlab%
\else%
\verttex{\epsfvlabsize}{\epsfysize}%
{\hbox to \epsfysize{\hss\box\epsfvertlab\hss}}%
\fi%
\ifvoid\epsfhorlab%
\else%
\scott=\epsfxsize%
\advance\scott by \epsfvlabsize%
\rlap{\vtop{\hrule height0pt\hbox to \scott{\hss\box\epsfhorlab\hss}}}%
\fi%
}
%
%  Place the figure labels
%
\def\epsfsetover{\ifvoid\epsffiglab\else\box\epsffiglab\fi}
%
%  Old stuff
%
\newread\epsffilein    % file to \read
\newif\ifepsffileok    % continue looking for the bounding box?
\newif\ifepsfbbfound   % success?
\newif\ifepsfverbose   % report what you're making?
\newdimen\epsfxsize    % horizontal size after scaling
\newdimen\epsfysize    % vertical size after scaling
\newdimen\epsftsize    % horizontal size before scaling
\newdimen\epsfrsize    % vertical size before scaling
\newdimen\epsftmp      % register for arithmetic manipulation
\newdimen\pspoints     % conversion factor
%
\pspoints=1bp          % Adobe points are `big'
\epsfxsize=0pt         % Default value, means `use natural size'
\epsfysize=0pt         % ditto
%
% Original version
%
\def\epsfbox#1{
   \ifvoid\epsfvertlab%
   \else\epsfvlabsize=\ht\epsfvertlab \advance\epsfvlabsize by \dp\epsfvertlab\fi%
   \leavevmode\global\def\epsfllx{72}\global\def\epsflly{72}%
   \global\def\epsfurx{540}\global\def\epsfury{720}%
   \def\lbracket{[}\def\testit{#1}\ifx\testit\lbracket
   \let\next=\epsfgetlitbb\else\let\next=\epsfnormal\fi\next{#1}}%
%
\def\epsfgetlitbb#1#2 #3 #4 #5]#6{\epsfgrab #2 #3 #4 #5 .\\%
   \epsfsetgraph{#6}}%
%
\def\epsfnormal#1{\epsfgetbb{#1}\epsfsetgraph{#1}}%
%
% Special version that set labels over the figure
%
\def\nepsfbox#1{
   \ifvoid\epsfvertlab%
   \else\epsfvlabsize=\ht\epsfvertlab \advance\epsfvlabsize by \dp\epsfvertlab\fi%
   \leavevmode\global\def\epsfllx{72}\global\def\epsflly{72}%
   \global\def\epsfurx{540}\global\def\epsfury{720}%
   \def\lbracket{[}\def\testit{#1}\ifx\testit\lbracket
   \let\next=\epsfgetlitbbo\else\let\next=\epsfnormalo\fi\next{#1}}%
%
\def\epsfgetlitbbo#1#2 #3 #4 #5]#6{\epsfgrab #2 #3 #4 #5 .\\%
   \epsfsetgrapho{#6}}%
%
\def\epsfnormalo#1{\epsfgetbb{#1}\epsfsetgrapho{#1}}%
%
\def\epsfgetbb#1{%
%
%   The first thing we need to do is to open the
%   PostScript file, if possible.
%
\openin\epsffilein=#1
\ifeof\epsffilein\errmessage{I couldn't open #1, will ignore it}\else
%
%   Okay, we got it. Now we'll scan lines until we find one that doesn't
%   start with %. We're looking for the bounding box comment.
%
   {\epsffileoktrue \chardef\other=12
    \def\do##1{\catcode`##1=\other}\dospecials \catcode`\ =10
    \loop
       \read\epsffilein to \epsffileline
       \ifeof\epsffilein\epsffileokfalse\else
%
%   We check to see if the first character is a % sign;
%   if not, we stop reading (unless the line was entirely blank);
%   if so, we look further and stop only if the line begins with
%   `%%BoundingBox:'.
%
          \expandafter\epsfaux\epsffileline:. \\%
       \fi
   \ifepsffileok\repeat
   \ifepsfbbfound\else
    \ifepsfverbose\message{No bounding box comment in #1; using defaults}\fi\fi
   }\closein\epsffilein\fi}%
%
%   Now we have to calculate the scale and offset values to use.
%   First we compute the natural sizes.
%
\def\epsfsetgraph#1{%
   \epsfrsize=\epsfury\pspoints
   \advance\epsfrsize by-\epsflly\pspoints
   \epsftsize=\epsfurx\pspoints
   \advance\epsftsize by-\epsfllx\pspoints
%
%   If `epsfxsize' is 0, we default to the natural size of the picture.
%   Otherwise we scale the graph to be \epsfxsize wide.
%
   \epsfxsize\epsfsize\epsftsize\epsfrsize
   \ifnum\epsfxsize=0 \ifnum\epsfysize=0
      \epsfxsize=\epsftsize \epsfysize=\epsfrsize
%
%   We have a sticky problem here:  TeX doesn't do floating point arithmetic!
%   Our goal is to compute y = rx/t. The following loop does this reasonably
%   fast, with an error of at most about 16 sp (about 1/4000 pt).
% 
     \else\epsftmp=\epsftsize \divide\epsftmp\epsfrsize
       \epsfxsize=\epsfysize \multiply\epsfxsize\epsftmp
       \multiply\epsftmp\epsfrsize \advance\epsftsize-\epsftmp
       \epsftmp=\epsfysize
       \loop \advance\epsftsize\epsftsize \divide\epsftmp 2
       \ifnum\epsftmp>0
          \ifnum\epsftsize<\epsfrsize\else
             \advance\epsftsize-\epsfrsize \advance\epsfxsize\epsftmp \fi
       \repeat
     \fi
   \else\epsftmp=\epsfrsize \divide\epsftmp\epsftsize
     \epsfysize=\epsfxsize \multiply\epsfysize\epsftmp   
     \multiply\epsftmp\epsftsize \advance\epsfrsize-\epsftmp
     \epsftmp=\epsfxsize
     \loop \advance\epsfrsize\epsfrsize \divide\epsftmp 2
     \ifnum\epsftmp>0
        \ifnum\epsfrsize<\epsftsize\else
           \advance\epsfrsize-\epsftsize \advance\epsfysize\epsftmp \fi
     \repeat     
   \fi
%
%  Finally, we make the vbox and stick in a \special that dvips can parse.
%
   \ifepsfverbose\message{#1: width=\the\epsfxsize, height=\the\epsfysize}\fi
   \epsftmp=10\epsfxsize \divide\epsftmp\pspoints
   \epsfsetlab%
   \ifincfigs%
     \vbox to\epsfysize{\vfil\hbox to\epsfxsize{%
        \special{PSfile=#1 llx=\epsfllx\space lly=\epsflly\space
            urx=\epsfurx\space ury=\epsfury\space rwi=\number\epsftmp}%
        \epsfsetover\hfil}}%
   \else%
     \epsfsetover%
     \vbox to\epsfysize{\hrule\vss\hbox to\epsfxsize{\vrule height
                        \epsfysize\hfil\vrule}\vss\hrule}%
   \fi%
\epsfxsize=0pt\epsfysize=0pt}%

%
% Version which sets labels over the graph
%

\def\epsfsetgrapho#1{%
   \epsfrsize=\epsfury\pspoints
   \advance\epsfrsize by-\epsflly\pspoints
   \epsftsize=\epsfurx\pspoints
   \advance\epsftsize by-\epsfllx\pspoints
%
%   If `epsfxsize' is 0, we default to the natural size of the picture.
%   Otherwise we scale the graph to be \epsfxsize wide.
%
   \epsfxsize\epsfsize\epsftsize\epsfrsize
   \ifnum\epsfxsize=0 \ifnum\epsfysize=0
      \epsfxsize=\epsftsize \epsfysize=\epsfrsize
%
%   We have a sticky problem here:  TeX doesn't do floating point arithmetic!
%   Our goal is to compute y = rx/t. The following loop does this reasonably
%   fast, with an error of at most about 16 sp (about 1/4000 pt).
% 
     \else\epsftmp=\epsftsize \divide\epsftmp\epsfrsize
       \epsfxsize=\epsfysize \multiply\epsfxsize\epsftmp
       \multiply\epsftmp\epsfrsize \advance\epsftsize-\epsftmp
       \epsftmp=\epsfysize
       \loop \advance\epsftsize\epsftsize \divide\epsftmp 2
       \ifnum\epsftmp>0
          \ifnum\epsftsize<\epsfrsize\else
             \advance\epsftsize-\epsfrsize \advance\epsfxsize\epsftmp \fi
       \repeat
     \fi
   \else\epsftmp=\epsfrsize \divide\epsftmp\epsftsize
     \epsfysize=\epsfxsize \multiply\epsfysize\epsftmp   
     \multiply\epsftmp\epsftsize \advance\epsfrsize-\epsftmp
     \epsftmp=\epsfxsize
     \loop \advance\epsfrsize\epsfrsize \divide\epsftmp 2
     \ifnum\epsftmp>0
        \ifnum\epsfrsize<\epsftsize\else
           \advance\epsfrsize-\epsftsize \advance\epsfysize\epsftmp \fi
     \repeat     
   \fi
%
%  Finally, we make the vbox and stick in a \special that dvips can parse.
%
   \ifepsfverbose\message{#1: width=\the\epsfxsize, height=\the\epsfysize}\fi
   \epsftmp=10\epsfxsize \divide\epsftmp\pspoints
%   \epsfsetlab%
   \ifincfigs%
     \vbox to\epsfysize{\vfil\hbox to\epsfxsize{%
        \special{PSfile=#1 llx=\epsfllx\space lly=\epsflly\space
            urx=\epsfurx\space ury=\epsfury\space rwi=\number\epsftmp}%
        \epsfsetover\epsfsetlab\hfil}}%
   \else%
     \epsfsetover%
     \vbox to\epsfysize{\hrule\vss\hbox to\epsfxsize{\vrule height
                        \epsfysize\hfil\vrule}\vss\hrule}%
   \fi%
\epsfxsize=0pt\epsfysize=0pt}%

%
%   We still need to define the tricky \epsfaux macro. This requires
%   a couple of magic constants for comparison purposes.
%
{\catcode`\%=12 \global\let\epsfpercent=%\global\def\epsfbblit{%BoundingBox}}%
%
%   So we're ready to check for `%BoundingBox:' and to grab the
%   values if they are found.
%
\long\def\epsfaux#1#2:#3\\{\ifx#1\epsfpercent
   \def\testit{#2}\ifx\testit\epsfbblit
      \epsfgrab #3 . . . \\%
      \epsffileokfalse
      \global\epsfbbfoundtrue
   \fi\else\ifx#1\par\else\epsffileokfalse\fi\fi}%
%
%   Here we grab the values and stuff them in the appropriate definitions.
%
\def\epsfgrab #1 #2 #3 #4 #5\\{%
   \global\def\epsfllx{#1}\ifx\epsfllx\empty
      \epsfgrab #2 #3 #4 #5 .\\\else
   \global\def\epsflly{#2}%
   \global\def\epsfurx{#3}\global\def\epsfury{#4}\fi}%
%
%   We default the epsfsize macro.
%
\def\epsfsize#1#2{\epsfxsize}
%
%   Finally, another definition for compatibility with older macros.
%
\let\epsffile=\epsfbox
%
%  End of epsf macros
%
%=============================================================================
%
%  Macros for defining symbols in text
%
\def\ifspace{\ifcat\issp.\else~\fi}
\def\tspace{\futurelet\issp\ifspace}
\def\a{({\it a\kern 1pt})\tspace}
\def\b{({\it b\kern 1pt})\tspace}
\def\c{({\it c\kern 1pt})\tspace}
\def\d{({\it d\kern 1pt})\tspace}
\def\e{({\it e\kern 1pt})\tspace}
\def\f{({\it f\kern 1pt})\tspace}
\def\g{({\it g\kern 1pt})\tspace}
\def\h{({\it h\kern 1pt})\tspace}
\def\i{({\it i\kern 1pt})\tspace}
\def\j{({\it j\kern 1pt})\tspace}
\def\abc#1{({\it #1\kern 1pt})\tspace}

\newcount\ndots
\def\drawline#1#2{\raise 2.5pt\vbox{\hrule width #1pt height #2pt}}
\def\spacce#1{\hskip #1pt}
%
%.... Line types
%
\def\solid{\hbox{\drawline{24}{.5}} }
\def\bdash{\hbox{\drawline{4}{.5}\spacce{2}}}
\def\dashed{\hbox{\bdash\bdash\bdash\bdash\nobreak} }
\def\bldash{\hbox{\drawline{6}{.5}\spacce{2}}}
\def\ldashed{\hbox{\bldash\bldash\bldash\nobreak} }
\def\bdot{\hbox{\drawline{1}{.5}\spacce{2}}}
\def\dotted{\hbox{\leaders\bdot\hskip 24pt\nobreak} }
\def\chndash{\hbox{\drawline{8.5}{.5}\spacce{2}\drawline{3}{.5}\spacce{2}\drawline{8.5}{.5}}\nobreak }
\def\chndot{\hbox{\drawline{9.5}{.5}\spacce{2}\drawline{1}{.5}\spacce{2}\drawline{9.5}{.5}}\nobreak }
\def\chndotdot{\hbox{\drawline{8}{.5}\spacce{2}\drawline{1}{.5}\spacce{2}\drawline{1}{.5}\spacce{2}\drawline{8}{.5}}\nobreak}
%
%.... Symbols
%
\def\trian{\raise 1.25pt\hbox{$\scriptscriptstyle\triangle$}\nobreak\ }
\def\solidtrian{\raise 1.25pt \hbox to 3bp{\special{" newpath  0 0 moveto 3 0 lineto 1.5 2.598 lineto closepath fill}\hfill}\nobreak\ }
\def\circle{$\circ$\nobreak\ }
\def\scircle{$\bullet$\nobreak\ }
\def\diam{$\diamond$\nobreak\ }
\def\solidcircle{$\bullet$\nobreak\ }
\def\smalltriangle{$\scriptstyle\triangle\textstyle$\nobreak\ }
\def\smallplus{$\scriptstyle + \textstyle$\nobreak\ }
\def\smalltimes{$\scriptstyle\times\textstyle$\nobreak\ }
\def\smallnabla{$\scriptstyle\nabla\textstyle$\nobreak\ }
\def\square{${\vcenter{\hrule height .4pt 
              \hbox{\vrule width .4pt height 3pt \kern 3pt \vrule width .4pt}
   	      \hrule height .4pt}}$\nobreak\ }
\def\solidsquare{${\vcenter{\hrule height 3pt width 3pt}}$\nobreak\ }
\def\plus{\raise 1.6pt \hbox{$\scriptscriptstyle +$}\nobreak\ }
\def\x{\raise 1.25pt \hbox{$\scriptscriptstyle \times$}\nobreak\ }
%
%.... Lines with symbols
%
\def\linecirc{\hbox{\drawline{24}{.5}\kern -14.5pt$\bullet$\kern 12pt}}
\def\dashplus{\hbox{\dashed\kern -19pt\plus\kern 8pt}}
%
%.... Legend macros
%
\def\legendtable#1{\vbox{\baselineskip=10pt\tabskip=0pt\let\\=\cr\halign{\hfil##\hskip 3pt&##\hfil\cr#1\crcr}}}
\def\lllegend#1 #2 #3{\figlab {#1} {#2} {\legendtable{#3}}}
\def\lrlegend#1 #2 #3{\figlab {#1} {#2} {\llap{\legendtable{#3}}}}
\def\ullegend#1 #2 #3{\figlab {#1} {#2} {\vtop{\hrule height 0pt\legendtable{#3}}}}
\def\urlegend#1 #2 #3{\figlab {#1} {#2} {\llap{\vtop{\hrule height 0pt\legendtable{#3}}}}}
%
%  duplex mode
%
\def\twosided{\ifTeXtures\else\special{ps::duplexon}\fi}
\def\onesided{\ifTeXtures\else\special{ps::duplexoff}\fi}
%
%  macros for putting scales on figures
%
\newdimen\xorigon
\newdimen\yorigon
\newdimen\scaleval
\newdimen\scaleorigon

\def\setxscale#1 #2 #3 #4 #5 {%
	\xorigon=#1\yorigon=#3%
	\scaleval=#2\advance\scaleval by -\xorigon%
	\tempdimen=#5 pt\advance\tempdimen by -#4pt%
	\divide\tempdimen by 1000%
	\divide\scaleval by \tempdimen%
	\scaleorigon=-#4pt\divide\scaleorigon by 1000%
	\multiply\scaleorigon by \scaleval}
\def\xtickup#1 #2{\tempdimen=#1pt\divide\tempdimen by 1000%
	\multiply\tempdimen by \scaleval\advance\tempdimen by \scaleorigon%
	\advance\tempdimen by \xorigon%
	\figlab {\tempdimen} {\yorigon} {\vbox {\hbox to 0pt{\hss #2\hss}%
		\baselineskip=8pt\lineskiplimit=-5pt%
		\hbox to 0pt{\hss \vrule height 3pt\hss}}}}
\def\xtickdown#1 #2{\tempdimen=#1pt\divide\tempdimen by 1000%
	\multiply\tempdimen by \scaleval\advance\tempdimen by \scaleorigon%
	\advance\tempdimen by \xorigon%
	\figlab {\tempdimen} {\yorigon} {\vbox to 0pt {\hbox to 0pt{\hss \vrule height 3pt\hss}%
		\nointerlineskip\vskip 3pt%
		\hbox to 0pt{\hss #2\hss}\vss}}}
%
%  Just a place holder
%
\def\nofig#1#2{\leavevmode{\vbox {\hrule \hbox to #1{\vrule height #2 \hfill \vrule} \hrule}} }

%%%%%%%%%%%%%%%%%%%%%%%%%%%%%%%%%%%%%%%%%%%%%%%%%%%%%%%%%%%%%%%%%%%%%%%%%%%%%%%
%
%  Macros: 
%
%  S. Scott Collis
%
%  Written: 9-5-95
%
%  Revised: 1-29-96
%
%%%%%%%%%%%%%%%%%%%%%%%%%%%%%%%%%%%%%%%%%%%%%%%%%%%%%%%%%%%%%%%%%%%%%%%%%%%%%%%
%
%  Change thesis to report for the technical report
%
\newcommand{\thesis}{thesis}
%
% LaTeX shortcuts
%
%\renewcommand{\S}{section\ }		% This lets you change section command
\newcommand{\eqn}[1]{(\ref{#1})}	% Handy reference to equations
\renewcommand{\epsilon}{\varepsilon}	% I like the other epsilon
\newcommand{\enp}{\epsilon}		% Nonparallel epsilon
%
\newcommand{\refs}{[{\bf ?}]\marginpar{$\Longleftarrow$}}
\newcommand{\eton}{\ensuremath{e^N}}
\newcommand{\Rgas}{{\cal R}}
\newcommand{\comma}{{\, ,}}
\newcommand{\period}{{\, .}}

% For units of measure
\newcommand\dynpercm{\nobreak\mbox{$\;$dynes\,cm$^{-1}$}}
\newcommand\cmpermin{\nobreak\mbox{$\;$cm\,min$^{-1}$}}

% Various bold symbols
\providecommand\bnabla{\boldsymbol{\nabla}}
\providecommand\bcdot{\boldsymbol{\cdot}}
\newcommand\biS{\boldsymbol{S}}
\newcommand\etb{\boldsymbol{\eta}}

% For multiletter symbols
\newcommand\Real{\mbox{Re}} % cf plain TeX's \Re and Reynolds number
\newcommand\Imag{\mbox{Im}} % cf plain TeX's \Im
\newcommand\Man{\mbox{\textit{M}}}  % Mach number
\newcommand\Ren{\mbox{\textit{Re}}}  % Reynolds number
\newcommand\Prn{\mbox{\textit{Pr}}}  % Prandtl number, cf TeX's \Pr product
\newcommand\Pen{\mbox{\textit{Pe}}}  % Peclet number

\newcommand\Ai{\mbox{Ai}}            % Airy function
\newcommand\Bi{\mbox{Bi}}            % Airy function

% For sans serif characters:
% The following macros are setup in JFM.cls for sans-serif fonts in text
% and math.  If you use these macros in your article, the required fonts
% will be substitued when you article is typeset by the typesetter.
%
% \textsfi, \mathsfi   : sans-serif slanted
% \textsfb, \mathsfb   : sans-serif bold
% \textsfbi, \mathsfbi : sans-serif bold slanted (doesnt exist in CM fonts)
%
% For san-serif roman use \textsf and \mathsf as normal.
%
\newcommand\ssC{\mathsf{C}}    % for sans serif C
\newcommand\sfsP{\mathsfi{P}}  % for sans serif sloping P
\newcommand\slsQ{\mathsfbi{Q}} % for sans serif bold-sloping Q

% Hat position
\newcommand\hatp{\skew3\hat{p}}      % p with hat
\newcommand\hatR{\skew3\hat{R}}      % R with hat
\newcommand\hatRR{\skew3\hat{\hatR}} % R with 2 hats
\newcommand\doubletildesigma{\skew2\tilde{\skew2\tilde{\Sigma}}}
%       italic Sigma with double tilde

% array strut to make delimiters come out right size both ends
\newsavebox{\astrutbox}
\sbox{\astrutbox}{\rule[-5pt]{0pt}{20pt}}
\newcommand{\astrut}{\usebox{\astrutbox}}

\newcommand\GaPQ{\ensuremath{G_a(P,Q)}}
\newcommand\GsPQ{\ensuremath{G_s(P,Q)}}
\newcommand\p{\ensuremath{\partial}}
\newcommand\tti{\ensuremath{\rightarrow\infty}}
\newcommand\kgd{\ensuremath{k\gamma d}}
\newcommand\shalf{\ensuremath{{\scriptstyle\frac{1}{2}}}}
\newcommand\sh{\ensuremath{^{\shalf}}}
\newcommand\smh{\ensuremath{^{-\shalf}}}
\newcommand\squart{\ensuremath{{\textstyle\frac{1}{4}}}}
\newcommand\thalf{\ensuremath{{\textstyle\frac{1}{2}}}}
\newcommand\Gat{\ensuremath{\widetilde{G_a}}}
\newcommand\ttz{\ensuremath{\rightarrow 0}}
\newcommand\ndq{\ensuremath{\frac{\mbox{$\partial$}}{\mbox{$\partial$} n_q}}}
\newcommand\sumjm{\ensuremath{\sum_{j=1}^{M}}}
\newcommand\pvi{\ensuremath{\int_0^{\infty}%
  \mskip \ifCUPmtlplainloaded -30mu\else -33mu\fi -\quad}}

\newcommand\etal{\mbox{\textit{et al.}}}
\newcommand\etc{etc.\xspace}
\newcommand\eg{e.g.\xspace}
\newcommand\ie{i.e.\xspace}

%\newtheorem{lemma}{Lemma}
%\newtheorem{corollary}{Corollary}

\newcommand{\bU}{{{\boldsymbol U}}}
\newcommand{\tbU}{{\tilde{\boldsymbol U}}}
\newcommand{\hbU}{{\hat{\boldsymbol U}}}
\newcommand{\bvbar}{{\overline{\boldsymbol v}}}
\newcommand{\pbar}{{\overline{p}}}
\newcommand{\bF}{{\boldsymbol F}}
\newcommand{\bq}{{\boldsymbol q}}
\newcommand{\bg}{{\boldsymbol g}}
\newcommand{\hbq}{{\hat{\boldsymbol q}}}
\newcommand{\bphi}{{{\boldsymbol \phi}}}
\newcommand{\bPhi}{{{\boldsymbol \Phi}}}
\newcommand{\hphi}{{\hat{\phi}}}
\newcommand{\bw}{{\boldsymbol w}}
\newcommand{\bv}{{\boldsymbol v}}
\newcommand{\tbv}{{\tilde{\boldsymbol v}}}
\newcommand{\hbv}{{\hat{\boldsymbol v}}}
\newcommand{\tp}{{\tilde{p}}}
\newcommand{\hp}{{\hat{p}}}
\newcommand{\bL}{{\mathcal L}}
\newcommand{\tbL}{{\tilde{\mathcal L}}}
\newcommand{\bj}{{\boldsymbol j}}
\newcommand{\bI}{{\bf I}}
\newcommand{\bsigma}{{\boldsymbol\sigma}}
\newcommand{\tbsigma}{{\tilde{\boldsymbol\sigma}}}
\newcommand{\bS}{{\bf S}}
\newcommand{\tbS}{{{\tilde{\bf S}}}}
\newcommand{\bs}{{\boldsymbol s}}
\newcommand{\bx}{{\boldsymbol x}}
\newcommand{\hbs}{{\hat{\boldsymbol s}}}
\newcommand{\tbs}{{{\tilde{\boldsymbol s}}}}
\newcommand{\Div}{{\mbox{div\,}}}
\newcommand{\Def}{{\mbox{def\,}}}
\newcommand{\Grad}{{\mbox{grad\,}}}
\newcommand{\bG}{{\boldsymbol G}}
\newcommand{\bA}{{\boldsymbol A}}
\newcommand{\bB}{{\boldsymbol B}}
\newcommand{\bC}{{\boldsymbol C}}
\newcommand{\bD}{{\boldsymbol D}}
\newcommand{\bE}{{\boldsymbol E}}
\newcommand{\bb}{{\boldsymbol b}}

\newcommand{\who}[1] { \widehat {\overline {#1}} }
\newcommand{\ol} { \overline }
\newcommand{\wh}{ \widehat }
\newcommand{\ou}{ \overline u }
\newcommand{\op}{ \overline p }
\newcommand{\odelta}{ \overline \Delta}
\newcommand{\oDelta}{ \overline \Delta}

\newcommand{\os} {\overline S}
\newcommand{\oS} {\overline S}

\newcommand{\pdt}[1] { \frac{\partial{#1}}{\partial t}}
\newcommand{\pdi}[2] { \frac{\partial{#1}}{\partial x_{#2}} }
\newcommand{\pdl}[2] { \frac{\partial}{\partial x_{#2}} \left({#1} \right) }
\newcommand{\pdone}[1] { \frac{\partial{#1}}{\partial x_{1}}}
\newcommand{\pdtwo}[1] { \frac{\partial{#1}}{\partial x_{2}} }
\newcommand{\pdthr}[1] { \frac{\partial{#1}}{\partial x_{3}} }
\newcommand{\pdoone} {\frac{\partial}{\partial x_{1}} }
\newcommand{\pdotwo} {\frac{\partial}{\partial x_{2}} }
\newcommand{\pdothr} {\frac{\partial}{\partial x_{3}} }
\newcommand{\pdd}[1] 
{\frac{\partial^{2}{#1}}{{\partial x_{j}}{ \partial x_{j}}} }

\newcommand{\onehalf} {\frac{1}{2}}

\newcommand{\into} { \int_\Omega }
\newcommand{\intw} { \int_{\Gamma_w} }

\newcommand{\intt} { \int_{t_0}^{t_0+T} }
\newcommand{\dto} { dt \, d\Omega}
\newcommand{\dtw} { dt \, d\Gamma}
\newcommand{\dom} { \, d\Omega }

\newcommand{\hu}{\hat u}
\newcommand{\hv}{\hat v}
\newcommand{\hw}{\hat w}


\begin{document}

\title{Notes on PSE and Adjoint PSE Code Development and Validation}
\author{S. Scott Collis}
\maketitle

\section{Validation of PSE}

\begin{enumerate}
\item I took the Blasius profile at $R=580$ with $\alpha=0.179$ and computed
the temporal growth-rate.  The result is $\omega = 6.517833\times 10^{-2} +
\iota 1.425171\times 10^{-3}$ which is in very good agreement with Grosh \&
Orszag (1997).
\item However, in comparing linear PSE to Bertolotti, I have found that my
nonparallel growth-rates are too low.  This must mean that I have messed up
the vertical component of velocity.  Note that when using the same base flow,
the LPSE and HLNS were in excellent agreement which indicates that the
base-flow is the culprit.
\item Currently the vertical velocity is given by
%
\begin{equation}
  v = \frac{1}{2\sqrt{R_0 l_s x}} \left( \eta f' + f \right)
\end{equation}
%
where
%
\begin{equation}
  \eta = \frac{l_s y}{\sqrt{l_s x / R_0}}
\end{equation}
%
and the Blasius equation is given by
%
\begin{equation}
  f''' + 0.5 f f'' = 0
\end{equation} 
%
The mistake was in $v$.  The equation should read
%
\begin{equation}
  v = \frac{1}{2\sqrt{R_0 l_s x}} \left( \eta f' - f \right)
\end{equation}
%
With this, I get results that closely match Bertolotti.  I have also computed
the mean-flow divergence and it is on the order of $1\times 10^{-6}$.  Note
that I still have a difference in the peak amplitudes depending on which
normalization condition I use.  This is particularly concerning since my
$u_{max}$ normalized results don't agree perfectly with Bertolotti.  Instead,
my $E_k$ norm results appear to be better?  I think that this error in $v$ may
have also been the cause of some of my difficulties in getting a starting
solution for the MDF.  I might go back and try to update $\alpha_{00}$ now
that I have fixed things.

\end{enumerate}

Before doing that I have implemented the normalization condition that
Bertolotti uses
%
\begin{equation}
  \frac{ \int_0^\infty \left( \hat u^* \hat u_{,x}  \right) \ dy }
       { \int_0^\infty \left( \hat u^* \hat u  \right) \ dy} = 0
\end{equation}
%
which is basically just the streamwise component of the disturbance kinetic
energy normalization.  My results with this are very close to the full $E_k$
normalization and the LPSE results at $F=86$ seem to match Bertolotti's
results well (even though I'm not really clear on which normalization that he
used in his figure 5.1.  He indicates that $u_{max}$ was used which doesn't
agree very well with my results.)

In comparing the {\em nonlinear} results, it seems that my forcing function
was not correct.  Upon review, I discovered that I was computing the
streamwise derivative of the solution incorrectly.  Upon fixing this, my
results are in better agreement with Bertolotti's however, they are still not
identical?  I also have tremendous difficulty in converging the solutions.  I
think that I will implement the phase locking that Bertolotti uses.  In this
case, only $\alpha_{01}$ is iterated on.  All the rest of the modes wavenumber
are integer multiples of the fundamental.  I have done runs at lower initial
amplitude $A_{01} = 0.1\%$ and convergence is decent (10-12 iterations) and
the solutions look qualitatively correct.  The number of iterations did not
change substantially when I switched between phase locked and non locked.
There was a slight difference in the computed $\alpha$ from the non-locked and
locked runs.  There was also a more significant transient in both $\alpha$ and
$\gamma$ when the wavenumbers (and hence the phase speeds) were not locked.
Howver, there was not a significant difference in the amplitude evolution.

It seems that the majority of my difficulties in convergence are due to the
formation of higher harmonics.  For consistency, I have swithed to
Bertolotti's convergence tolerance which is basically looks at the magnitude
of the changes in $\alpha$, {\em not} the relative changes.  Bertolotti does
claim that convergence is faster when using trapezoidal.  This statement may
be misleading -- he may be refering to spatial convergence and not convergence
of the iteration scheme.  I will try converging to $1\times 10^{-6}$ to see if
the solution is still okay -- still seems okay.

Is it possible that aliasing is messing up the results for larger amplitudes.
Perhaps I should try to increase the number of temporal modes just to make
sure.  I increase the number of modes to 24 with no noticable difference.

I have tried to go back and put in the $\alpha_{00}$ term.  Now it seems to
converge just fine with this in place.  However, the solution is somewhat
different -- presumably more accurate since the exponential growth of the 00
mode is now accounted for exactly.  But, the iteration eventually fails to
converge.

\section{Nonlinear PSE}

{\bf Note} that I initially started these calculations (1-6) using a
fundamental that was the result of a previous LPSE run.  Because of this, the
value of $u_{max} = 0.782248$ which means that to get the actual amplitude, I
have scaled the value used in the inflow file by this number.

With a little help from Christopher Hill I was finally able to get the
nonlinear PSE working.  The problem was completely due to the
mean-flow-distortion (MFD) mode which must be handled differently from the
other periodic modes.  There are two primary difficulties: 1) getting a
reasonable starting profile for the MFD and, 2) updating $\alpha_{00}$.  To
get a good starting profile, Hill suggests that you solve an inhomogeneous
problem for the 00 mode where the forcing function comes from the nonlinear
term and you turn off the streamwise derivatives (I was already doing this).
The trick is that you have to set $\alpha_i$ for this mode to be slightly
negative.  Hill suggests that $\alpha_i = -1.0\times10^{-3}$ based on
$\delta^*$.  I find that a more negative value gives better results.
Basically you are trying to account for the streamwise derivative using
$\alpha$.  However, after the first station, you set $\alpha_{00}=0$ and
absorb all the changes in the MDF profile using the streamwise derivative.  In
this way, there is no iteration on $\alpha_{00}$.

For the harmonic, I started a LPSE using the inhomogeneous solution and one
using the parallel flow eigenfuction.  The resulting $alpha_{02}(x)$ where in
excellent agreement except for a slight transient when using the inhomogeneous
solution.  Note that the eigenfuction caused only a very slight transient.
Note, however, that the evolution of this mode is very different in NPSE.

I will try placing all of the initial profiles in inflow files -- this worked
just fine.  I am now trying to compute an initial profile only for the MDF
(run2).  For the other mode, I do it like I used to.  This gives a {\em much}
cleaner solution for the higher mode!  So I should only precompute the mode
shape for the MDF term.  The others can be made on the fly -- run2 is my
reference solution for low amplitudes.

The initial run was done with $A_{01} = 0.078\%$, this immediately initiated
modes $0-2$.  In run3, I have increased the initial amplitude, based on
$u_{max}$ to $A_{01} = 0.78\%$ which initiated modes $0-4$.  In this run, the
MDF was great enough that the stability characteristics of the higher modes
was dramatically influence.  Similar to Bertolotti, I get that the fundamental
continues to grow downstream of the linear Branch II neutral point.  Note that
the transient associated with the initiation of new modes appears to get worse
the higher the harmonic.

\begin{figure}
\centering \epsfxsize=4.5in 
\sethlabel{$\tilde x$}
\setvlabel{$E_k$}
\nepsfbox{amp.r2.eps}
\caption {Amplitude evolution from nonlinear PSE: $R_0=200$, $F=150$,
$A_{01}=0.1\%$. \label{f:amp.r2}}
\end{figure}

\begin{figure}
\centering \epsfxsize=4.5in 
\sethlabel{$\tilde x$}
\setvlabel{$E_k$}
\nepsfbox{amp.r3.eps}
\caption {Amplitude evolution from nonlinear PSE: $R_0=200$, $F=150$,
$A_{01}=0.078\%$. \label{f:amp.r3}}
\end{figure}

\begin{figure}
\centering \epsfxsize=4.5in 
\sethlabel{$\tilde x$}
\setvlabel{$-\gamma$}
\nepsfbox{gr.r3.eps}
\caption {Evolution of modal growth rates from nonlinear PSE: $R_0=200$,
$F=150$, $A_{01}=0.78\%$. \label{f:gr.r3}}
\end{figure}

\begin{figure}
\centering \epsfxsize=4.5in 
\sethlabel{$\tilde x$}
\setvlabel{$\alpha$}
\nepsfbox{alp.r3.eps}
\caption {Evolution of wavenumber from nonlinear PSE: $R_0=200$,
$F=150$, $A_{01}=0.78\%$. \label{f:alp.r3}}
\end{figure}

\begin{figure}
\centering \epsfxsize=4.5in 
\sethlabel{$\tilde x$}
\setvlabel{$-\gamma$}
\nepsfbox{gr.amp.eps}
\caption {Effect of initial fundamental amplitude on fundamental growth-rate
from nonlinear PSE: $R_0=200$, $F=150$. \label{f:gr.amp.r3}}
\end{figure}

In run4, I chose an initial amplitude of $A_{01}=0.39\%$.  This cases a slight
increase in instability -- something is wrong with the MDF mode in the run?
It seems that the MDF mode was messed up in the initial condition.  I have
rerun this case and have gotten correct results.

In run5, I am investigating the use of different initial conditions for the
higher harmonics.  I switched back to $A_{01} = 0.078\%$ and used the
eigenfunction from parallel theory for the first harmonic with an amplitude of
$a_{02} = 4.5\times10^{-6}$ which was selected to match the inhomogeneous
local solution.  Surprisingly, this leads to a larger transient then using the
local inhomogeneous solution -- perhaps due to inaccurate phase.  It looks
like it may be best to just use the inhomogeneous solution to initialize the
higher harmonics -- it certainly is more convenient.  In cases where you want
more control over the initial condition, you can allways provide the higher
harmonics but you have to live with the transient that you get.

In run6, I have increased the initial fundamental amplitude to $A_{01} =
1.56\%$.  It indeed does get difficult to converge.  I have observed that the
higher the initial amplitude, not only does the growth rate increase, but the
wavenumber decreases.  So, the fundamental grows faster and is longer due to
nonlinear effects.  For this case, the growth rate of the fundamental goes out
the roof and eventually blows up.

In run7, I used the LST eigenfunction as the fundamental initial profile using
a corrected initial $A_{01}=0.2\%$ based on $u_{max}$.  As expected the use of
the LST eigenfunction causes a transient that is most noticable in the
fundamental growth-rate.  Compared to my prior runs, this run most closely
matches run4 in which the corrected initial amplitude is 0.39.

I think that I will do a run8 in which the initial amplitude of the LST
fundamental eigenfunction is set to 0.4 in order to match with run4 above.
The runs are in close agreement with the run8 fundamental a little higher, as
expected.  Interestingly, the transient for the 02 mode is greater for run4 as
compared to run8.  However, the fundamental is cleaner for run4, especially
noticable in the growth-rate.

\begin{figure}
\centering \epsfxsize=4.5in 
\sethlabel{$\tilde x$}
\setvlabel{$E_k$}
\nepsfbox{amp.r8.eps}
\caption {Amplitude evolution from nonlinear PSE: $R_0=200$, $F=150$,
$A_{01}=0.4\%$. \label{f:amp.r8}}
\end{figure}

\begin{figure}
\centering \epsfxsize=4.5in 
\sethlabel{$\tilde x$}
\setvlabel{$-\gamma_{01}$}
\nepsfbox{gr.r48.eps}
\caption {Comparison of fundamental growth-rates from nonlinear PSE:
$R_0=200$, $F=150$, $A_{01} \approx 0.4\%$. \label{f:gr.r48}}
\end{figure}

\subsection{Comparison to Bertolotti}

Bertolotti starts with $R=400$ using $F=86$.  He includes six Fourier modes in
time and three initial amplitudes: 0.20\%, 0.25\%, and 0.30\%.  I have
constructed a Blasius field using $Nx=200$ from $R=400$ to $R=1000$ with
$R_0=400$.

\section{Discrete Adjoint Solution}

\begin{enumerate}
\item It took quite a bit of effort to get a clean discrete adjoint method
\item The main impediment was that I didn't account for the mapping implied by
the Chebyschev method!  However, this mapping does introduce a problem in that
the weight function of the Chebyshev polynomials is given by $w(x) =
(1-\eta^2)^{-1/2}$ where $\eta \in [-1,1]$.  Note that the weight goes to zero
at the boundaries which when included in the discrete inner product means that
the boundary values must be zero.  However, the surface pressure is not zero
and this leads to a problem.  I corrected it by using an artificial, nonzero
weight on the wall boundary.
\item With the above fix, the discrete solution is smooth in all quantities
except for approximately 5 nodes in pressure near the wall when using the
$\partial v/\partial y=0$ wall BC.  The error in the wall pressure appears to
be about 4\%.
\item I went to great lengths to statically elliminate the Neumann boundary
conditions.  If you don't do this the discrete adjoint wall pressure is 0 when
using both the $\partial v/\partial y=0$ wall BC and the wall normal momentum
equation.
\item If I use the wall normal momentum equation to determine the wall
pressure, there is a significant oscillation in $v$ and the pressure also is
in error.  This is better if I set $\partial v/\partial y=0$, although there
is still a small error in the wall pressure.  Again, things are better if you
statically elliminate the Neumann boundary condition.
\item I do think that a lot of these problems are coming from the zero
boundary weights for Chebyschev.
\item The finite difference discrete adjoint does not work as well.  Although
the mapping in this case remains finite at the boundaries. My guess is that I
am seeing the classic problem of an overconstrained pressure.  If I were to
use a staggered mesh, I would anticipate that many if not all of these
problems would be reduced.
\item The bottomline is that I can compute the discrete adjoint using the
spectral method.  The eigenvalues are identical to the regular problem and the
eigenfunctions should be highly orthogonal.
\item There is a tendency for the discrete adjoint solutions to exhibit more
node-to-node oscillation than the regular equations.  I think that this is a
property (undesireable) of the discrete opperators used.
\item I had been using the $y$-parameter $y_{max}=80$, $y_{str}=0.01$.  This
stretching is too much -- it causes oscillations in the mean flow derivative.
I backed off to $y_{str}=0.05$ and things look good.  The adjoint may not be
as good since the gradients near the wall are higher.  Now try $y_{str}=0.025$
-- this is better for the adjoint.
\end{enumerate} 

\section{Linear PSE validation}

I have revised the {\tt blasius.f90} to base the input and output on the
Blasius length scale $\tilde\delta_0 = \sqrt{\tilde\nu \tilde x_0 / \tilde
U_e}$ so that the reference Reynolds number is given by $R_0 = \tilde U_e
\tilde\delta_0 / \tilde\nu$.  To get things in terms of the displacement
thickness you need to use the relationship $\delta^*/\delta = 1.7208$.  The
frequency parameter is defined as
%
\begin{equation}
  F = \frac{\tilde\omega \tilde\delta_0}{\tilde U_e} 10^6 / R_0
\end{equation}
%
Note that the nondimensional inputs to the PSE code are $R_0$ and $\omega = F
R_0 10^{-6}$.  In this case the reference length-scale is $\tilde\delta_0$,
the velocity scale is $\tilde U_e$ and the time-scale is
$\tilde\delta_0/\tilde U_e$.  If you want to convert the $x$ coordinate back
to the local Blasius coordinate, then you multiply by the quantity
$\sqrt{R_0/x}$ which I typically do in an {\tt awk} script.

To test the code I have run the case $F=150$ from $R=300$ to $R=600$.  From my
PSE run I get the first neutral point at $R_I=370$ and the second at $R_{II} =
520$.

The growth rate is defined as
%
\begin{equation}
  \gamma = \frac{1}{Q}\frac{\partial Q}{\partial x}
\end{equation}
%
Bertolotti defines the following notation: $u,v$ are real, physical velocity
components, $u',v'$ are the root-mean-square of physics velocities, and $\hat
u,\hat v$ are the complex velocity profiles.  He bases his normalization
criteria on $u'_{max}$ where the maximum is taken in the wall normal
direction.  He says that the amplitude based on $u'_{max}$ is proportional to
the vorticity strength in the critial layer making this an important quantity
for secondary stability analysis.  The complex wavenumber\footnote{Note that
Bertolotti's $a$ is equal to $i$ times mine.} $a$ is then given by
%
\begin{equation}
  a(x) = -i \frac{1}{\hat u(x,y_m)} \frac{\partial\hat u(x,y_m)} {\partial x}
\end{equation}
%
so that the growth-rate $\gamma = -Im(a)$ and the wavenumber is $\alpha =
Re(a)$.

Bertolotti always set $\partial a/\partial x = 0$.  I find that there is no
advantage to doing so.

I have implemented a second-order accurate time advancement scheme based on
midpoint rule.  The convergence is a little slower and it does poorly in the
transient -- 2-$\Delta$ waves are formed.  If I first start with backward
Euler and then switch to midpoint the solution is good.  However, for the
limited resolution studies that I have done, there is no advantage of using a
second order scheme -- even with only 4 points per wave.  However, if you
intend to resolve the transient then there could be an advantage to a second
order scheme.  However, you really have to be careful since I don't think that
the transient with PSE is accurate anyway!  Note that Bertolotti starts with
Backward Euler and then smoothly changes to Trapezoidal and that he finds that
with Trapezoidal $a$ converges faster.  I do not observe a speedup in
convergence with trapezoidal although I haven't tried to smoothly change to
it.

I have experimented with the wall pressure boundary condition.  It seems like
the best alternative is to enforce continuity at the wall.  If you use the
wall normal momentum equation there can be a pressure jump at the wall.  In
body-fitted coordinates the continuity equation is given by
%
\begin{equation}
  c_1 u_{,x} + c_2 v + v_{,y} + w_{,z} = 0
\end{equation}
%
where
%
\begin{equation}
  c_1 = \frac{1}{1 + \kappa y} \qquad c_2 = \kappa c_1
\end{equation}
%
and $\kappa$ is the local surface curvature.  Typically due to the no-slip
condition $u_{,x}$ and $w_{,z}$ will be zero.  However, if one uses a
linearized roughness boundary condition this will not be the case.  For
generality, these terms are included when solving for $v$ at the wall using
this approach.

\begin{figure}
\centering \epsfxsize=4.5in 
\sethlabel{$R$}
\setvlabel{$-\gamma$}
\nepsfbox{f150.eps}
\caption {PSE growth rate based on disturbance kinetic energy for Blasius
boundary layer at $F=150$. \label{f:f150}}
\end{figure}

I have implemented two different strategies for normalizing the PSE shape
funtion.  The figure is as used by Bertolotti
%
\begin{equation}
  \frac{1}{\hat u_{max}}\frac{\partial \hat u_{max}}{\partial x} = 0
\end{equation}
%
The second is from Herbert
%
\begin{equation}
  \frac{ \int_0^\infty \left( \hat u^* \hat u_{,x} + \hat v^* \hat v_{,x} + 
         \hat w^* \hat w_{,x} \right) \ dy }
       { \int_0^\infty \left( \hat u^* \hat u + \hat v^* \hat v + 
         \hat w^* \hat w \right) \ dy} = 0
\end{equation}
%
Overall the code works better with the second normalization.  However, there
are some differences in the growth-rate results using both codes.  Futhermore,
the wavenumber tends to get corrupted when using the $u_{max}$ normalization.
This may be due to the rather poor method that I am currently using to find
the maximum.  I have implemented a B-Spline method for determining the maximum
which should be much more accurate.  This has fixed the oscillations in $a$
that I had previously observed.

However, I still get slightly different growth rates when using the different
normalizations.  Perhaps the transient is worse when using the $u_{max}$
normalization.  As the $R$ increases the differences between the two
normalizations vanishes.  I have performed a run starting at $R_0=200$ still
at $F=150$ and (as seen in figure \ref{f:f150}) it looks like the differences
are {\em not} due to the transient but are an artifact of the normalization.
Note that to plot the code results on a constent scale, the $x$ coordinate
must be multiplied by $\sqrt{R_0/x}$ and the complex wavenumber by
$\sqrt{x/R_0}$.  Figure \ref{f:gamma} shows the effect of PSE normalization on
the predicted growth-rate using two measures.  I think that it would be a
worth performing a comparison of LNS and PSE results to determine which of the
methods is the best.

\begin{figure}
\centering \epsfxsize=4.5in 
\sethlabel{$R$}
\setvlabel{$\gamma$}
\nepsfbox{gamma.eps}
\caption {Comparison of PSE growth-rate results based on $E$ and $\hat
u_{max}$ for Blasius boundary layer at $F=150$. \label{f:gamma}}
\end{figure}

\section{Comparison to HLNS}

Take the $R_0=300$ to $R=600$ case and perform a forced HLNS solve on it.  In
order to do this, I need to make a new $x$-mesh in {\tt blasius.f90} so that
there is refinement near the inflow.  I am going to use the same technique
used by Streett in {tt hdir.f90}.  The input parameters are $x_1$, $x_2$, and
$\Delta x_1$.  Basically, $x_1$ and $x_2$ are breakpoints.  The mesh begins
with spacing $\Delta x_1$ and is uniform till $x_1$.  The mesh then transforms
to a new spacing between $x_1$ and $x_2$ and is uniform at the new spacing
after $x_2$.  Between $x_1$ and $x_2$ the mesh spacing changes quadratically.

I have modified {\tt blasius} to support the nonuniform $x$-mesh.  With the
new mesh, I have rerun PSE and have verified that the results are mesh
independent.  Now use the inflow data and mean flow to compute a HLNS
solution.  In {\tt hdir} I implemented a new inflow boundary condition {\tt
ibci=3} that reads an eigenfunction file {\tt inflow.dat} and places that
complex profile on the inflow boundary.  My first HLNS is on the same mesh as
the PSE so that it is very poorly resolved.  I had an error in the inflow
boundary condition which has now been fixed.  Okay, things appear to be
working now.  First, check the growth-rate computation and compare against PSE
results.  It is of the same magnitude, but considerably more oscillatory.  

I have made several HLNS runs on a $Ny=32$ mesh to experiment with streamwise
resolution.  The domain covers the same range as that in the PSE
calculation. I am currently unsure as to whether the buffer domain is working
properly.  It appears that I am getting some significant reflections?  I have
tried runs with $Nx=200,400,800,1000$ to see the effect of streamwise
resolution.  Things have gotten better with $Nx=100$ although I still have
significant oscillations in my HLNS growth rate.  I bet that these are due to
poor resolution in the buffer domain so I have increased the resolution to
$Nx=1500$ to test this hypothesis.  I may want to try the damping function
used by Fasel which they claim reduces reflections.  The results with
$Nx=1500$ are virtually the same as the $Nx=1000$ case.  I am now trying to
increase $amuc = 15$ instead of 8.  It is virtually identical!  Now trying an
increase of amu2c.

run1, amuc = 8,  amu2c=20
run2, amuc = 15, amu2c=20
run3, amuc = 15, amu2c=80

Again, this gives no significant improvement!

Looking at the latest fields shows that they are relatively clean.  I might be
able to improve the results by refining near the inflow boundary since there
is a significant transient there.  All previous runs where performed with $fx1
= 0.5$ and $bx1=0.2$.  I am switching this to $fx1=0.25$ and $bx1=0.2$ to
evaluated the effect of inflow resolution -- no significant difference.  

On Howard's airfoil, I ended up using $amuc=6$, $amu2c=20$ and the results
there were {\em much} better!  I'll give these a shot here.  Note that I used
very high resolution in $y$ for Howard's airfoil.

Now try increasing vertical resolution from $Ny=32$ to $Ny=48$ to see if this
has an influence.  The results seem to be better up to $x=1000$ but then get
worse -- probably due to my buffer layer (this observation is for the
growth-rate based on $u_{max}$, for dke the $y$ resolution made no significant
change.  Switching back to $amuc=6$ and $amu2c=20$ just like that used for
Howard.  The solution is not really any difference.  How about making the
buffer domain a little longer -- try changing $amuloc$ from 1600 to 1400.
With the longer buffer the results are basically the same.

The final thing that I can try is to use a PSE profile on the inflow.  This
should dramatically reduce the initial transient.  For example, I could put a
profile from the $R=200$ run onto the inflow of the $R=300$ mesh.  If the
oscillations are an artifact of the inflow forcing, then I should be able to
reduce them.  In the case of Howard's airfoil, I was doing receptivity
simultions which perhaps lead to less oscillations in the growth-rate.  I
could try placing a suction/blowing slot at, say, $b_{loc}=400$ with a
$b_{con}=100$.  This should be fairly well resolved on the current mesh.  It
could be the case that the parallel-flow eigenfunction is such a poor
approximation of the nonparallel flow natural mode that it excites a host of
poorly damped modes.  This may particularly the case since the inflow forcing
occurs for all $y$.  

Trying suction/blowing for the $Ny=32$ case.

Things work much better for Howard's airfoil case?  Return to this case and
verify that I get the same answer.  I have run this case on the SGI and I
get results that are identical to the original Cray version.

I have restructured the Blasius run based on the successful Howard run.  This
includes: increasing $y_{max}$, making the buffer $2 \lambda_{ts}$, using the
same buffer parameters, increasing $Ny=64$, using $40$ nodes per streamwise
wavelength where $\lambda_{ts} \approx 80$.  In the domain from $200$ to
$1800$ there are approximately $20$ periods so I am using 800 points.  Note
that on Howard's airfoil, I really only had about 15 points per wavelength.
For this same resolution, I could get away with 300 points in the Blasius
case.

In order to make a high signal-to-noise ratio, I have used a suction/blowing
source at 400 with width of 40 units.  This is roughly similar to Howard's
case except that relative to Branch I, the source location is further
upstream.  If you want the maximum effective amplitude then I should really
put the source near Branch I (which is at approximately $x=680$).  However,
there has to be some projection onto the most unstable mode so that I should
get a reasonable response.  Note that at this $F$ there are only about 8
wavelengths of the instability wave within the unstable region.

If you really want to get a low transient solution, then I guess that you had
better put a PSE shapefunction on the domain inflow.  If I start PSE at
$R_0=200$ then I can extract the profile at $x=450$ an use it in a calculation
starting at $R_0=300$.

I have implemented the buffer function used by Meitz \& Fasel which is defined
by
%
\begin{equation}
  c(\xi) = \exp\left(-\frac{\xi^4}{10}\right)\left( 1 - \xi^50 \right)^4
\end{equation}
%
where $\xi = (x-x_B)/(x_{max}-x_B)$.  They claim that this buffer function
reduces the relection of waves from the junction at the upstream end of the
buffer domain.  If it works, this seems like a good idea since I am having
extreme resolution difficulties due to reflections at the junction.  The first
run that I have done with this buffer is for the $Ny=32$ case with boundary
forcing since I have data to compare against using the original buffer.  The
results look identical to the original buffer function.  However, I guess that
I should try to do a calculation with fewer points in $x$ to see if the new
function is easier to resolve.  To test this, I am trying $Nx=300$ -- the
results seem no better than what I had with the old buffer function.  Now
trying $Nx=600$ -- this is better but still not as good as the $Nx=1000$
case. From 1000 to 1500 the difference is small.

It really seems to boil down to signal-to-noise ratio.  It appears that you
may have a problem when the amplitude of the instability wave is small.  For
example, if you force at the boundary far upstream of the NP then the primary
instability wave may be too small relative to errors caused by the
buffer-layer for an accurate measure of the growth-rate.  I'm going back to
the $R_0=300$ case and placing a PSE shape-function on the inflow boundary.
To get the inflow condition I ran a PSE run starting at $R_0=200$ to $x=450$.
I then modified the {\tt blasius.f90} code to allow for an arbitrary $R_1$
given a reference $R_0$.  Thus, I keep $R_0=200$ but set $R_1=300$.  Of course
the resolution for the HLNS run is increased in $x$ and I used the same
outflow boundary location $R_2=600$.  
%
\begin{enumerate}

\item The result is fairly good agreement with PSE with very little inflow
transient, but I still get oscillations in the growth rate.  Originally I set
$x_b=1600$.

\item On a second run I am setting $x_b=1500$ to see if a more gentle buffer
transition will help reduce oscillations -- no difference.

\item Try increaseing $amu2c = 200$ on run3 -- basically no difference in the
growth-rate, although the influence of the buffer is felt further upstream.

\item Now trying $amuc=0$ on run4 -- this actually worked fiarly well,
although there was a buildup of energy at the outflow, the upstream influence
was similar.  The energy build-up may be due to the pressure specification.
Otherwise, it looks like the wave progresses fiarly smoothly out the domain.

\item It looks like the main problem is due to the specification of pressure on
the outflow boundary at $Ny-1$.  To investigate this, run5 sets the pressure
at $Ny/2+1$.  The result is generally the same, but with some $2\Delta$ wave
in the solution.

\item Going back to setting pressure at $Ny-1$, try using $ivbc=2$.  Using
continuity at $y_{max}$ gives considerably better results as shown in figures
\ref{f:cont} and \ref{f:cont_umax}.  This indicates that the trouble is likely
due to the upper boundary condition.  Right now, the upper boundary is rather
close to the wall at $Ny=120$.

\begin{figure}
\centering \epsfxsize=4.5in 
\sethlabel{$\tilde x$}
\setvlabel{$\gamma$}
\nepsfbox{cont.eps}
\caption {Comparison of PSE growth-rate results based on $E$ for Blasius
boundary layer at $F=150$. \label{f:cont}}
\end{figure}

\begin{figure}
\centering \epsfxsize=4.5in 
\sethlabel{$\tilde x$}
\setvlabel{$\gamma$}
\nepsfbox{cont_umax.eps}
\caption {Comparison of PSE growth-rate results based on $u_{max}$ for Blasius
boundary layer at $F=150$. \label{f:cont_umax}}
\end{figure}

\item I think that the best solution is to move the upper boundary farther
away from the wall.  Try $y_{max}=200$, $y_{str}=0.08$ on run7.  By moving the
upper boundary further away, the error due to setting the pressure to zero at
the corner point should be reduced.  I am also using continuity on the upper
boundary which was shown in run6 to give improved results.  The taller mesh
does give improved results as seen in figures \ref{f:ymax} and
\ref{f:ymax_umax}, although I think that there is a trade-off between
resolution and boundary location.

\begin{figure}
\centering \epsfxsize=4.5in 
\sethlabel{$\tilde x$}
\setvlabel{$\gamma$}
\nepsfbox{gr_ke_y.eps}
\caption {Comparison of PSE growth-rate results based on $E$ for Blasius
boundary layer at $F=150$. \label{f:ymax}}
\end{figure}

\begin{figure}
\centering \epsfxsize=4.5in 
\sethlabel{$\tilde x$}
\setvlabel{$\gamma$}
\nepsfbox{gr_umax_y.eps}
\caption {Comparison of PSE growth-rate results based on $u_{max}$ for Blasius
boundary layer at $F=150$. \label{f:ymax_umax}}
\end{figure}

\item Now try $N_y = 400$, $y_{str}=0.05$. Things get worse.  I guess that the
$y$-resolution is not good enough.  

\item Try increasing $N_y = 48$ to see if the problem is resolution
related. This didn't fix the problem.  It seems that the problem is due to the
floating mean pressure.  Fixing the pressure on the outflow does not seem to
constrain the upstream pressure to the correct value.  In fact, the inflow
pressure at $y_{max}$ is nowhere hear zero.  Instead it is a complex constant
approximately equal to $-0.12 - 0.56i$. Setting the pressure only on the
inflow didn't work.  Now I'm trying setting pressure to zero all along the top
boundary.  This should be a good approximation...  Maybe the buffer does
generate a pressure gradient which is why the pressure cannot be constant in
the far-field?

\item Okay, I am setting the pressure to zero on the inflow.  Using $\partial
p/\partial y=0$ on the interior and outflow and am now using continuity on the
ouflow.

\end{enumerate}

It does look like the Fasel buffer function may work better than Streett's.  I
tried to turn on a little bit of $amu2c=1$ and the results are worse than
without it. What if I arbitrarily increase the amplitude of the inflow wave.
Since everything is linear -- this should cause no difference in the results.
I may also want to increase the $x$ resolution just to make sure that is the
culprit.  Right now, the HLNS amplitude is a little high -- usually this
occurs when the $x$-resolution is low.  I increased $N_x=1000$ to see whether
the amplitude and growth-rate converge to the PSE values.  There was no
significant difference in the solution.

If you ramp down the streamwise viscous terms in-order to make the equations
parabolic near the outflow, it seems to me that $\partial p/\partial x$ must
increase in the buffer to compensate.  The other option is to increase the
$y$-viscous terms so that there is a new balance.  Of course, doing so will be
quite tricky in terms of setting the factor on these terms.  If you could
estimate $\partial^2 u/\partial x^2$ then you could just add this term on as a
source term.  This is possible if you know an approximate value for $\alpha$,
say from PSE analysis.  Given $\alpha$ which is assumed to be constant, then
you can write
%
\begin{equation}
  u = \hat u e^{i\alpha x}
\end{equation}
%
then the first derivative is given by
%
\begin{equation}
  u_{,x} = \hat u_{,x} e^{i\alpha x} + i \alpha \hat u e^{i\alpha x}
\end{equation}
%
which can be solved for
%
\begin{equation}
  \hat u_{,x} e^{i\alpha x} =  u_{,x} - i \alpha \hat u e^{i\alpha x}
\end{equation}
%
The second derivative, after using the above expression, is then given by
%
\begin{equation}
  u_{,xx} = \hat u_{,xx} e^{i\alpha x} + 2i\alpha u_{,x} + \alpha^2 u
\end{equation}
%
Thus, if $\hat u_{,xx}$ is assumed small, then the second derivative can be
approximated by
%
\begin{equation}
  u_{,xx} \approx 2i\alpha u_{,x} + \alpha^2 u
\end{equation}
%
I have used this in the code and it really doesn't do much.  I don't
understand why this doesn't fix the problem.  I guess that I could include the
actual distribution for $\alpha$, but I really don't think that this will
help.  By making the buffer a little longer $x_b=1400$ instead of $x_b=1500$,
I was able to improve the results slightly, but the extent of the upstream
influence did not change.  

Eureka, the problem is in the streamwise pressure gradient, of course!  You
have to do the following
%
\begin{equation}
  p_{,x} = \hat p_{,x} e^{i\alpha x} + i\alpha p
\end{equation}
%
To make the PSE parabolic, the term $\hat p_{,x}$ is assumed to be zero.  If I
do the same thing in the HLNS code, {\bf I get an almost perfect outflow
boundary condition!!!}

\begin{figure}
\centering \epsfxsize=4.5in 
\sethlabel{$\tilde x$}
\setvlabel{$u_{max}$}
\nepsfbox{umax.eps}
\caption {Comparison of amplitudes based on $u_{max}$ for the Blasius
boundary layer at $F=150$. \label{f:umax}}
\end{figure}

\begin{figure}
\centering \epsfxsize=4.5in 
\sethlabel{$\tilde x$}
\setvlabel{$E_k$}
\nepsfbox{ke.eps}
\caption {Comparison of amplitudes based on $E_k$ for Blasius
boundary layer at $F=150$. \label{f:ke}}
\end{figure}

\begin{figure}
\centering \epsfxsize=4.5in 
\sethlabel{$\tilde x$}
\setvlabel{$\gamma$}
\nepsfbox{gr_umax.eps}
\caption {Comparison of growth-rate based on $u_{max}$ for the Blasius
boundary layer at $F=150$. \label{f:gr_umax}}
\end{figure}

\begin{figure}
\centering \epsfxsize=4.5in 
\sethlabel{$\tilde x$}
\setvlabel{$\gamma$}
\nepsfbox{gr_ke.eps}
\caption {Comparison of growth-rate based on $E_k$ for Blasius
boundary layer at $F=150$. \label{f:gr_ke}}
\end{figure}

The current results shown in figures \ref{f:umax} -- \ref{f:gr_ke}
deomonstrate the success of this boundary treatment.  Note that in these
results I have used the value of $\alpha = 0.07820 + i 0.000711024$ which is
the PSE predicted value at the outflow (using KE normalization).  It seems to
me, that the critical feature is to have the real part -- the imaginary part
is nearly zero anyway.  I have tried a new run using on the real part and the
results are shown in figure \ref{f:gr_umax_alphar} compared to the full
complex $\alpha$.  Clearly the results are not quite as good, although they
are still superior to the original buffer and the errors are localized near
the boundary.  Since it is relatively easy to estimate the real part of
$\alpha$ this may be a useful method.

\begin{figure}
\centering \epsfxsize=4.5in 
\sethlabel{$\tilde x$}
\setvlabel{$\gamma$}
\nepsfbox{gr_umax_alphar.eps}
\caption {Comparison of growth-rate based on $u_{max}$ for the Blasius
boundary layer at $F=150$: \solid real $\alpha$, \dashed is complex
$\alpha$.\label{f:gr_umax_alphar}}
\end{figure}

\begin{itemize}
\item Is it neccessary to include the viscous terms?  Maybe all I really need
to do is approximate the pressure gradient?
\item Will the solution quality be improved if I use the exact PSE $\alpha(x)$
distribution?  I tried this and the solution is virtually identical to using
just the outflow value of $\alpha$.
\end{itemize}

If you set $amu = 0$ then the HLNS code will really solve the PSE since the
second derivatives in the streamwise direction have been removed and the
pressure gradient is only given by the wave component.  However, it is a
little bit different since I am still requiring that $u$ represent the total
variation of the solution.  Therefore, I will still need sufficient resolution
to resolve the wave component of the solution.  The next step is to code the
equations similar to PSE so that the solution $\hat u$ is on the slowly
varying shape-function.  Then I should get very similar results to PSE.
However, the current method, up to trunction error, should be similar to PSE
since the same assumptions are made.  There should be only slight upstream
influence since both the pressure and viscous terms are parabolized.

This works very well, although there is a slight increase in $2\Delta$ that is
noticable in the growth-rates.  This is probably due to the difference in
numerical methods/accuracy between the two methods and the use of central
differencing in the HLNS code.  The upshot of this is that the HLNS results
with forced $\alpha$ are identical to the PSE results.  In other words, the
full LNS solution shows greater growth than the PSE.  It might be possible
that this growth is a result of a transient, but I don't think so.  The HLNS
growth-rates still have a slight oscillation to it, but is not at the same
mean value as the PSE.  There is a slight error in pressure on the inflow
boundary when using $\alpha(x)$ in the HLNS.  I'm not sure of the source of
this, but it doens't seem to affect the rest of the solution.

\section{Re-Validation of PSE/HLNS}

Using the corrected mean flow results and the new outflow buffer technique, I
quickly reran the PSE/HLNS comparisons that I had made previously at $F=150$.
These runs use the PSE shapefunction inflow profile.  It appears that the
solutions are in good agreement between the methods, although the general
conclusion is the same as above -- the growth rate with HLNS is slightly
higher than PSE.  I also tried a quick resolution study in $x$ to make sure
that $N_x=800$ is adequate.  I tried $N_x=600$ and the transient after the
inflow is stronger and the $2\delta$ waves at the outflow are larger in
amplitude.  For $N_x=1000$, the inflow transient is better than the $N_x=800$
case, and the $2\Delta$ near the outflow are smaller.  However, the
growth-rate is still larger with HLNS compared to PSE.  With $N_x=1200$ there
is no decernible difference with the $N_x=1000$ case, including the growth
rate curves.  The overall quality of the results is very good, however there
still is a slight transient in the HLNS growth rate, especially the growth
rate based on disturbance kinitic-energy.

\section{PSE validation revisited (8-4-99) }

\begin{enumerate}
\item As a sanity check, I have gone back to the parallel flow case to
validate my PSE and APSE.  First I looked at LST and compared continuous and
discrete PSE eigenfunction and they are in excellent agreement.

\item I did a 3d TS wave on a parallel Blasius boundary layer and LNS and PSE
are in perfect agreement.

\item I have returned to the Parabolic cylinder and have compared PSE and LNS
for $\beta=35$.  There are some differences which could be due to transients!
Otherwise the results are in good agreement.  It is very difficult to get good
inflow profiles that are transient free for PSE.  This makes it very difficult
to get clean growth rate curves near the upstream neutral point.  For the bump
case, I tried starting a PSE at $i=227$.  This gives very good results,
although the growth-rate predicted by PSE is a little larger than that from
LNS, especially near the location of maximum growthrate.  Could the LNS need
more resolution?  This should be established.

\item What if your run APSE backwards, get the adjoint shapefunction and use
this to project to a good PSE shapefunction?  Try this for the {\tt
pcyl\_new/beta=100} case.

\item The new buffer works well for crossflow instabilities on the parabolic
cylinder.  I tried {\tt amuc} of 8 and 20 with no discernable difference.

\item I have compared all the HLNS codes and they are in good order.

\item I have compared the {\tt src} and {\tt dev} versions of the PSE codes
and they are also in good order.  The main differences lie in the solvers {\tt
asolver.f90, lst.f90, tlst.f90, dasolver.f90} and {\tt pse.f90}.  Currently
the adjoint solver in {\tt src} uses the KE normalization which is clearly
wrong.  The adjoint solver in {\tt dev} uses the PSE alpha and does one solve
to get the shape functions.  This is less wrong, but still not right of
course.  In this case $\alpha = -\tilde\alpha$ but $\hat\bj_{,x} \ne 0$.

\end{enumerate}

\section{Adjoint PSE}

\begin{enumerate}
\item I have implemented an iteration into the {\tt asolver.f90} code in {\tt
dev}.  This uses the following normalization for 
\end{enumerate}

\section{Nonlinear HLNS}

\section{Boundary conditions}

Currently I use the following far-field boundary condition for LST:
$u=v=w=p=0$.  I think that a more reasonable condition would be zero traction.
To set this BC requires that 
\[ u_{,y} = w_{,y} = 0 \]
\[ p - \frac{1}{Re} v_{,y} = 0 \]   
I implemented this by requiring that $p = 0$ and $v_{,y}=0$.  This works fine
for LST.

\end{document}
