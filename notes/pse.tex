%========================================================================
%
%  PSE Users guide
%
%  S. Collis
%
%  8-2-99
%========================================================================
\documentclass[10pt]{article}
\usepackage{fullpage}
\usepackage{xspace}
\usepackage{supertabular}
\usepackage{longtable}

%\usepackage{lucidbry}

\newcommand{\pse}{\textsf{PSE}\xspace}
\newcommand{\lst}{\textsf{LST}\xspace}
\renewcommand{\ast}{\textsf{AST}\xspace}

\begin{document}

\title{\pse Users Guide}
\author{S. Scott Collis \\
Rice University \\
Houston, TX 77005-1892 }
\date{\today}
\maketitle

\section*{Coordinates and definitions}

\pse is written in a body fitted curvilinear coordinate system where $x$ is
the streamwise direction, $y$ is the wall normal direction, and $z$ is the
spanwise direction.  Time is denoted by $t$ and the temporal variation of
solutions is represented in the frequency domain where $\omega$ is the
fundamental frequency.  Similarly, a Fourier representation is also used in
$z$ with a fundamental wavenumber of $\beta$.  The velocity components are
represented by $u$, $v$, and $w$ in the streamwise, wall normal, and spanwise
directions, respectively.

\section*{\pse input file}

The input file for \pse is called {\tt pse.inp} and it is in Fortran90
namelist format.  The following table defines each parameter and the values
that they can take.

\setlongtables
\begin{longtable}[c]{|l|l|l|}
\caption {\pse Input Parameters \label{t:parm}} \\
\hline
\multicolumn{1}{|l}{\textbf{Parameter}} & 
\multicolumn{1}{|c|}{\textbf{Description}} &
\multicolumn{1}{c|}{\textbf{Notes}} \\ 
\hline
\endfirsthead
\multicolumn{3}{c}{Table \ref{t:parm}: Continued from pervious page} \\[1ex]
\hline
\multicolumn{1}{|l}{\textbf{Parameter}} & 
\multicolumn{1}{|c|}{\textbf{Description}} &
\multicolumn{1}{c|}{\textbf{Notes}} \\ 
\hline
\endhead
\hline
\multicolumn{3}{r}{continued on next page} \\[1ex]
\endfoot
\hline
\endlastfoot
%
{\tt ITYPE} & Execution mode & 
     0 : linear \pse \\
& &  1 : nonlinear \pse \\
& &  2 : adjoint \pse \\
& &  3 : discrete adjoint \pse \\
& & -1 : spatial \lst \\
& & -2 : temporal \lst \\
& & -3 : spatial \ast \\
& & -4 : temporal \ast \\
& & -5 : discrete temporal \ast \\
& & -6 : discrete spatial  \ast \\
{\tt NX} & Number of nodes in $x$ & only used with {\tt IMEAN}=0 \\
{\tt NY} & Number of nodes in $y$ & overrides mean flow {\tt NY} \\
{\tt NZ} & Number of modes in $z$ & \\
{\tt NT} & Number of modes in $t$ & \\
{\tt BETA} & Fundamental spanwise wavenumber & \\
{\tt OMEGA} & Fundamental frequency& \\
{\tt RE} & Reference Reynolds number & \\
{\tt ICURV} & Curvature flag & 0 : no curvature terms \\
& & 1 : curvature terms \\
{\tt IMEAN} & Meanflow type & 0 : parallel mean flow \\
& & 1 : boundary layer flow \\
& & 2 : Navier--Stokes flow \\
{\tt IVBC} & $v$-velocity BC on top boundary & 0: $v = 0$ \\
& & 1 : $\partial v/\partial y|_w = 0$ \\
{\tt IPWBC} & pressure BC on wall boundary & 0 : 
$\partial p/\partial y|_w = 0$ \\
& & 1 : wall normal momentum equation \\
& & 2 : continuity at wall \\
{\tt IPBC} & pressure BC on top boundary & 0 :  $p = 0$ \\
& & 1 : $\partial p/\partial y|_{y_max} = 0$ \\
{\tt MKZ} & Spanwise modes in initial condition & for nonlinear \pse \\
{\tt MKT} & Frequencies in initial condition & for nonlinear \pse \\
{\tt XMAX} & Length of domain in $x$ & only for {\tt IMEAN} = 0 \\
{\tt XS1} & Stretching parameter 1 & only for {\tt IMEAN} = 0 \\
{\tt XS2} & Stretching parameter 2 & only for {\tt IMEAN} = 0 \\
{\tt DX1} & Initial $\Delta x$ & only for Imean = 0 \\
{\tt Ymax} & Maximum $y$ & Should be several BL thicknesses \\
{\tt YSTR} & Stretching parameter & $ 0 < {\tt YSTR} < 1 $ \\
{\tt TOL} & Tolerance for \pse iterations & $\approx 1 \times 10^{-8}$ \\
{\tt IPFIX} & Pressure update for \pse & use 1 \\
{\tt NITER} & Number of \pse iterations& typically 5-50\\
{\tt IS} & Starting index for \pse & location for \lst \\
{\tt IE} & Ending index for \pse & location for \ast \\
{\tt NORM} & \pse normalization flag & 0 : Kinetic energy \\
& & 1 : $u_{max}$ \\
{\tt NINT} & Interpolation in {\tt NY} for $u_{max}$ evaluation & \\
{\tt INT} & Interpolation factor in {\tt NX} for field output & 
$< 1$ denotes no interpolation \\
{\tt SOR} & SOR parameter for \pse iteration & $1 \le {\tt SOR} < 2$ \\
{\tt PLOCK} & Phase lock flag & 0 : unlocked \\
& & 1 : phase locked \\
{\tt EPS} & Mode generation tolerance & $1\times 10^{-8}$ \\
{\tt ALPI} & Meanflow modification parameter & $-1\times 10^{-3}$ \\
\end{longtable}

\section*{Mean flow input files}

As indicated in Table \ref{t:parm} there are three options for the meanflow.
If running on a parallel baseflow, ${\tt IMEAN}=0$, then the code reads the
profile from {\tt profile.dat}.  For boundary layer base flows, ${\tt
IMEAN}=1$, the baseflow is read from {\tt bl\_sta.out}.  And the last option is
for a Navier--Stokes solution, ${\tt IMEAN}=2$ which is read from {\tt
field.dat}.

\section*{Disturbance initial profiles} \label{s:initial}

The disturbance profiles are read from {\tt inflow.???} where the extension is
used to denote the spanwise wavenumber and the temporal frequency.  The
fundamental is denoted with a 1, the first harmonic by 2, etc.  For example,
{\tt inflow.+12} is the fundmental $+$ mode in the span and the harmonic in
time.  Likewise {\tt inflow.+01} has no variation in the span and the
fundamental frequency, and {\tt inflow.+00} is the mean flow modification
term.

\section*{Output}

The primary \pse output is in the file {\tt pse.dat} which contains $\alpha$
and other quantities of interest during the \pse calculation.  The file {\tt
grow.dat} contains various measures of the disturbance amplitude as well as
associated growth rates.  The file {\tt pro.dat} contains the \pse
shapefunctions at the last computed station.

The code also outputs the entire computed fields (possibly interpolated in the
$x$ direction depending on the value of {\tt INT} ) for each combination of
spanwise and temporal modes using the same indexing notation described in
\ref{s:initial} but with the basename {\tt field.???}.

To convert a {\tt field.???} file to Plot3D format use the program {\tt
mkqr8}.
\end{document}
